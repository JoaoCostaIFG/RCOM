\documentclass{report}

\usepackage{graphicx} % Add pictures to your document
\usepackage{float}

\begin{document}

\title{\huge{\textbf{RCOM TP1}} \\ Turma 3 // TODO grupo \\ RCOM 2020/2021 \\ MIEIC FEUP}
\author{João de Jesus Costa \\ \texttt{up201806560} \and
	João Lucas Silva Martins \\ \texttt{up201806436}}
\date{\today{}}

\begin{figure}[b]
	\centering
	\includegraphics[width=0.6\textwidth]{feup_logo.png}
\end{figure}
\maketitle{}

\tableofcontents{}
\newpage

\chapter{Sumário}
// TODO

\chapter{Introdução}
Este relatório incide sobre o projecto desenvolvido para a unidade curricular
de RCOM. Neste projecto foi pedido o desenvolvimento de uma aplicação, em
linguagem C, que permitisse o envio de ficheiros, entre dois computadores,
através de uma \textit{serial port}.

Além disso, esta aplicação devia estar organizada em camadas independentes
(discutidas mais à frente) e ser resistente a ruído e desconexão durante
o envio de informação.

\chapter{Arquitetura}

{\let\clearpage\relax \chapter{Estrutura do código}}

{\let\clearpage\relax \chapter{Casos de usos principais}}

\chapter{Protocolos}

\section{Protocolo de ligação lógica}

\section{Protocolo de aplicação}

\chapter{Validação}

\chapter{Eficiência do protocolo de ligação de dados}

\chapter{Conclusões}

\end{document}

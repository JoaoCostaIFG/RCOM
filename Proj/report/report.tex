\documentclass{report}

\usepackage{graphicx} % Add pictures to your document
\usepackage{float}
\usepackage{listings}
\usepackage{xcolor}
\lstset { %
    language=C,
    backgroundcolor=\color{black!5}, % set backgroundcolor
    basicstyle=\footnotesize,% basic font setting
}

\begin{document}

\title{\huge{\textbf{RCOM TP1}} \\ Turma 3 // Grupo 3 \\ RCOM 2020/2021 \\ MIEIC FEUP}
\author{João de Jesus Costa \\ \texttt{up201806560} \and
	João Lucas Silva Martins \\ \texttt{up201806436}}
\date{\today{}}

\begin{figure}[b]
	\centering
	\includegraphics[width=0.6\textwidth]{feup_logo.png}
\end{figure}
\maketitle{}

\tableofcontents{}
\newpage

\chapter{Sumário}
// TODO

\chapter{Introdução}
Este relatório incide sobre o projecto desenvolvido para a unidade curricular
de RCOM. Neste projecto foi pedido o desenvolvimento de uma aplicação, em
linguagem C, que permitisse o envio de ficheiros, entre dois computadores,
através de uma \textit{serial port}.

Além disso, esta aplicação devia estar organizada em camadas independentes
(discutidas mais à frente) e deve ser resistente a ruído e desconexão durante
o envio de informação.

// TODO Dizer o q cada chapter tem

\chapter{Arquitetura}

O código encontra-se dividido em duas partes principais: a camada de aplicação e
a camada de ligação de dados. Por sua vez, estas duas camadas dividem-se na sua
componente pública, que é utilizada por um agente externo (\textbf{interface}), e a sua
componente privada que integra as funções internas da camada.

{\let\clearpage\relax \chapter{Estrutura do código}}

\section{Camada de aplicação}

A API da camada de aplicação é implementada nos ficheiros \textit{app\_layer.h}
e \textit{app\_layer.c}. Para utilizar esta camada é necessário instanciar um objeto
\textbf{struct applicationLayer}.

\begin{lstlisting}
struct applicationLayer {
  int fd; /* file descriptor correspondente a porta serie */
  enum applicationStatus status; /* TRANSMITTER or RECEIVER */
  char file_name[256];           /* name of file to transmit (if any) */
  long file_size;                /* size of file to transmit (if any) */
  long chunksize;                /* tansmission chunksize */
};
\end{lstlisting}

A função \textbf{initAppLayer} inicia um objeto do tipo \textbf{struct applicationLayer}
e a sua correspondente \textbf{struct linkLayer}.
\begin{lstlisting}
void initAppLayer(struct applicationLayer *appLayer, int baudrate,
                  long chunksize);
\end{lstlisting}

A função \textbf{llopen} abre uma conexão na dada \textit{serial port}.
\begin{lstlisting}
int llopen(int porta, enum applicationStatus appStatus);
\end{lstlisting}

A função \textbf{llwrite} envia o dado \textit{buffer} através da ligação pré estabelicida.
\begin{lstlisting}
int llwrite(int fd, char *buffer, int length);
\end{lstlisting}

A função \textbf{llread} recebe um pacote da ligação pré-estabelecida para \textit{buffer}.
\begin{lstlisting}
int llread(int fd, char **buffer);
\end{lstlisting}

A função \textbf{llclose} fecha uma ligação anteriormente estabelecida.
\begin{lstlisting}
int llclose(int fd, enum applicationStatus appStatus);
\end{lstlisting}

A função \textbf{sendFile} lê e envia o ficheiro especificado pela camada de aplicação.
\begin{lstlisting}
int sendFile(struct applicationLayer *appLayer);
\end{lstlisting}

A função \textbf{sendFile} recebe o conteúdo de um ficheiro transmitido à camada de aplicação,
guardando-o em \textit{res}.
\begin{lstlisting}
int receiveFile(struct applicationLayer *appLayer, unsigned char **res);
\end{lstlisting}

A função \textbf{write\_file} cria um ficheiro com o conteúdo de \textit{file\_content}
e com nome especificado pela camada de aplicação.
\begin{lstlisting}
void write_file(struct applicationLayer *appLayer, unsigned char *file_content);
\end{lstlisting}

\section{Camada de ligação}

A API da camada de ligação é definida nos ficheiros \textit{data\_link.h} e
\textit{data\_link.c}. Para utilizar esta camada é necessário instanciar um objeto
do tipo \textbf{struct linkLayer}.

\begin{lstlisting}
struct linkLayer {
  char port[20];                 /* Dispositivo /dev/ttySx, x = 0, 1 */
  int baudRate;                  /* Velcidade de transmissao */
  unsigned int sequenceNumber;   /* Numero de sequencia da trama: 0, 1*/
  unsigned int timeout;          /* Valor do temporizador, e.g.: 1 sec */
  unsigned int numTransmissions; /* Numero de retransmissoes em caso de falha */
  vector *frame;
};
\end{lstlisting}

A função \textbf{initLinkLayer} inicializa uma camada de ligação e o seu respetivo \textbf{vector}.
\begin{lstlisting}
struct linkLayer initLinkLayer();
\end{lstlisting}

A função \textbf{initConnection} estabelece uma conexão do tipo indicado em \textit{isReceiver}
na camada de ligação dada.
\begin{lstlisting}
int initConnection(struct linkLayer *linkLayer, int fd, bool isReceiver);
\end{lstlisting}

A função \textbf{endConnection} fecha uma conexão do tipo indicado em \textit{isReceiver}
na camada de ligação dada.
\begin{lstlisting}
int endConnection(struct linkLayer *linkLayer, int fd, bool isReceiver);
\end{lstlisting}

A função \textbf{getFrame} recebe uma trama através de uma conexão pré-estabelecida.
\begin{lstlisting}
int getFrame(struct linkLayer *linkLayer, int fd, unsigned char **packet);
\end{lstlisting}

A função \textbf{getFrame} envia uma trama para uma conexão pré-estabelecida com o tamanho
\textit{len}.
\begin{lstlisting}
int sendFrame(struct linkLayer *linkLayer, int fd, unsigned char *packet,
              int len);
\end{lstlisting}

{\let\clearpage\relax \chapter{Casos de usos principais}}

A nossa solução para a proposta de trabalho contempla dois casos de uso: o envio de um ficheiro
como emissor e a receção do mesmo como recetor. Este modo de funcionamento deve ser referido
na chamada do programa através da flag \textit{-s RECEIVER} ou \textit{-s TRANSMITTER}.

Para o programa ser executado como recetor é obrigatório especificar o número da porta série a usar.



Se o programa for executado como emissor, será necessário especificar o caminho para o ficheiro a enviar,
em adição aos outros argumentos já enunciados anteriormente. Enquanto emissor, também pode ser útil referir
o \textit{chunksize} a usar.

\chapter{Protocolos}

\section{Protocolo de ligação lógica}

Este protocolo e responsável pela preparação da informação a enviar pela
\textit{serial port} e os mecanismos de recuperação de erros/falhas.

Este protocolo transmite informação através do uso de tramas. Existem três
tipos de tramas: informação, supervisão, e não numeradas. As tramas de
informação carregam os dados a enviar. As tramas de supervisão indicam
se um dada de trama de informação foi bem (ou mal) recebida. As tramas
não numeradas servem para implementar os mecanismos de inicio e conclusão
de uma conexão.

\section{Protocolo de aplicação}

\chapter{Validação}

\chapter{Eficiência do protocolo de ligação de dados}

\chapter{Conclusões}

\end{document}

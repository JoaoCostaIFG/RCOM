\documentclass[11pt]{report}

\usepackage{graphicx} % Add pictures to your document
\usepackage{float}
\usepackage{xcolor}
\usepackage{pgfplots}
\usepackage{sectsty}

\begin{document}

\title{\huge{\textbf{RCOM Lab2}} \\ Turma 3 // Grupo 3 \\ RCOM 2020/2021 \\ MIEIC FEUP}
\author{João de Jesus Costa \\ \texttt{up201806560} \and
	João Lucas Silva Martins \\ \texttt{up201806436}}
\date{\today{}}

\begin{figure}[b]
  \begin{center}
    \includegraphics[width=0.6\textwidth]{feup_logo.png}
  \end{center}
\end{figure}
\maketitle{}

\tableofcontents{}
\newpage

\chapter{Summary}

This report documents the project that was developed for the curricular unit
of RCOM (FEUP). We implemented a simple FTP download application and configured
a computer network with three computers, a commercial router and switch.

\chapter{Introduction}

In the first part of this report, we'll focus on the FTP download application
and its architecture.\\
In the second part of this report, we'll focus on the network configuration
and analyze it.

\chapter{Part 1 - Download application}

\section{Architecture}

The application works linearly. We start by opening a socket to the given
server (on the standard FTP port: 21) and authenticate the user. This is
the \textbf{control socket}.\\
After the authentication process, we enter passive mode (with the PASV
command) and open a second socket to the server on the port replied on
the PASV command. This is the \textbf{data socket}.

Once we have this two sockets open, we just need to request the file
we want to the control socket and read it from the data socket. If any
of the steps described fails, the execution is aborted and the user is
informed of the error.

\section{Download steps}
\begin{enumerate}
\item Parse the url and authentication information
\item Resolve the hostname
\item Open control socket to server
\item Authenticate
\item Enter passive mode
\item Open data socket to server
\item Request file on control socket
\item Read file from the data socket
\end{enumerate}

\chapter{Part 2 - Network configuration and analysis}

\chapter{Conclusions}

\chapter{References}

\chapter{Annexes}

\end{document}
